% Florian Sihler, 2022
% Licensed under MIT License
% https://opensource.org/licenses/MIT
\documentclass[parskip=half,english,numbers=noenddot,footnotes=nomultiple,oneside]{scrartcl}
\errorcontextlines99999

\usepackage[T1]{fontenc}
\usepackage[utf8]{inputenc}
\usepackage{babel}

\usepackage{microtype}
\usepackage[hidelinks]{hyperref}

\usepackage{lmodern}

\usepackage[prefix=]{xcolor-material}
\usepackage{listings}

\usepackage{lc-visualizer}

% working with texcsstyle did not work
\lstdefinelanguage{ltx}{
   literate={\\LC}{\textcolor{purple}{\bfseries\textbackslash LC}}3
   {\\RLC}{\textcolor{purple}{\textbackslash RLC}}4
   {\\lcDisableLambdaRanges}{\textcolor{purple}{\textbackslash lcDisableLambdaRanges}}{22}
   {\\lcEnableLambdaRanges}{\textcolor{purple}{\textbackslash lcEnableLambdaRanges}}{21}
   {\\tikzset}{\textcolor{purple}{\textbackslash tikzset}}8
   {\\lcSetColorMixin}{\textcolor{purple}{\textbackslash lcSetColorMixin}}{16}
   {\\lcSetColors}{\textcolor{purple}{\textbackslash lcSetColors}}{12}
   {\\lcDoNotShadeSameVariables}{\textcolor{purple}{\textbackslash lcDoNotShadeSameVariables}}{26}
   {\\lcDoShadeSameVariables}{\textcolor{purple}{\textbackslash lcDoShadeSameVariables}}{23}
   {\\lcDoNotStrutLine}{\textcolor{purple}{\textbackslash lcDoNotStrutLine}}{17}
   {\\lcDoStrutLine}{\textcolor{purple}{\textbackslash lcDoStrutLine}}{14}
   {\\lcCreateNewRewritingRule}{\textcolor{purple}{\textbackslash lcCreateNewRewritingRule}}{25}
   {\\lcMaximumParenthesisRangeDrawDepth}{\textcolor{purple}{\textbackslash lcMaximumParenthesisRangeDrawDepth}}{35}
   {\\lcSetDuplicateVariableMixin}{\textcolor{purple}{\textbackslash lcSetDuplicateVariableMixin}}{28}
   {\\reducemarks}{\textcolor{purple}{\textbackslash reducemarks}}{12}
   {\\lcMaximumRewritingRuleExpandDepth}{\textcolor{purple}{\textbackslash lcMaximumRewritingRuleExpandDepth}}{34}
   {\\lcEnableGlow}{\textcolor{purple}{\textbackslash lcEnableGlow}}{13}
   {\\mark}{\textcolor{purple}{\textbackslash mark}}5
   {\\a}{\textcolor{gray}{\textbackslash a}}2
   {\\b}{\textcolor{gray}{\textbackslash b}}2
   {\\c}{\textcolor{gray}{\textbackslash c}}2
   {\\cons}{\textcolor{orange}{\textbackslash cons}}5
   {\\f}{\textcolor{gray}{\textbackslash f}}2
   {\\x}{\textcolor{gray}{\textbackslash x}}2
   {\\n}{\textcolor{gray}{\textbackslash n}}2
   {\\y}{\textcolor{gray}{\textbackslash y}}2
   {\\.}{\textcolor{gray}{\textbackslash .}}2
   {\\centerline}{\textcolor{purple}{\textbackslash centerline}}{11}
   {\\Y}{\textcolor{purple}{\textbackslash Y}}2
   {\\equiv}{\textcolor{purple}{\textbackslash equiv}}6
   {\\Ycomb}{\textcolor{orange}{\textbackslash Ycomb}}6
   {\\ifelse}{\textcolor{orange}{\textbackslash ifelse}}7
   {\\num}{\textcolor{orange}{\textbackslash num}}4
   {\\mult}{\textcolor{orange}{\textbackslash mult}}5
   {\\pred}{\textcolor{orange}{\textbackslash pred}}5
   {\\isZero}{\textcolor{orange}{\textbackslash isZero}}7
   {\\\\}{\textcolor{teal}{\textbackslash \textbackslash}}2
}
\lstset{basicstyle=\ttfamily,keepspaces,breaklines,language=ltx}
\lstMakeShortInline|
\def\ltx#1{\lstinline/#1/}
\interfootnotelinepenalty=10000
\lcDoNotStrutLine

\usepackage{bm}
\lcSetLambda{\bm\lambda}

\title{The \texttt{lc-visualizer} package}
\author{%
	\texorpdfstring{Florian Sihler\medskip\\[-2mm]
		{\scriptsize\url{https://github.com/EagleoutIce/latex-lambda-calculus-visualizer}}%
	}{Florian Sihler}}
\date{Version v1.0 \textendash{} 2022/10/28}


\begin{document}
   \maketitle

	\texttt{lc-visualizer} is a simple package to get highlighting for lambda-calculus expressions.
   Issuing |\LC{(\\a.\\b.\a (\b \\b.\b \a)) a c}| yields the following:

   \LC{(\\a.\\b.\a (\b \\b.\b \a)) a c}

   There is also |\RLC|, a variant that does not do any catcode magic and that can be used inside other macros. So |\centerline{\RLC{\Y \equiv \\f\.\(\\x\.\f \(\x \x\)\) \(\\x\.\f \(\x \x\)\)}}| yields:

   \centerline{\RLC{\Y \equiv \\f\.\(\\x\.\f \(\x \x\)\) \(\\x\.\f \(\x \x\)\)}}

   \section{Shortcut and customization options}
      You can pass on multiple letters to a lambda-abstraction: |\LC{\\{a,b}.\b \a}| yields: \LC{\\{a,b}.\b \a}.

      With |\lcDisableLambdaRanges| you can disable the brackets drawn around to signal the abstraction ranges of lambdas locally (or re-enable with |\lcEnableLambdaRanges|):
      |{\lcDisableLambdaRanges\LC{((\\a.\a) \b \a)}}| yields: {\lcDisableLambdaRanges\LC{((\\a.\a) \b \a)}} (vs. \LC{((\\a.\a) \b \a)}).

      With |\lcMaximumParenthesisRangeDrawDepth| you can (similarly to the effects of the |\lcDisableLambdaRanges| macro) allow to draw brackets for parenthesis.
      However, being disabled by default you can no configure the maximum nesting-level to draw this brackets for.
      Writing |{\lcDisableLambdaRanges\lcMaximumParenthesisRangeDrawDepth{2}\LC{((\\a.\a) \b \a)}}| yields: {\lcDisableLambdaRanges\lcMaximumParenthesisRangeDrawDepth{2}\LC{((\\a.\a) \b \a)}} (vs. {\lcDisableLambdaRanges\LC{((\\a.\a) \b \a)}}).

      If you do not like the colors you can change their cycle locally with |\lcSetColors|: |{\lcSetColors{teal,lime}\LC{((\\a.\a) \\b.\b \a)}}| yields: {\lcSetColors{teal,lime}\LC{((\\a.\a) \\b.\b \a)}} (vs. \LC{((\\a.\a) \\b.\b \a)}). Furthermore, both |\LC| and |\RLC| have an optional argument to set the start index in the color-cycle (defaults to \(0\)):
      |\LC[3]{((\\a.\a) \\b.\b \a)}| yields: \LC[3]{((\\a.\a) \\b.\b \a)}. This can be used in combination with leftmost-outermost beta-reductions!

      With |\lcSetColorMixin| you can (locally) change the way colors are mixed in the background highlights (with |#1| being the actual color selected): |{\lcSetColorMixin{#1!70!white}\LC{((\\a.\a) \\b.\b \a)}}| yields the output: {\lcSetColorMixin{#1!70!white}\LC{((\\a.\a) \\b.\b \a)}} (vs. \LC{((\\a.\a) \\b.\b \a)}).

      The package will try its best to keep colors in sync and recognizable:
      |\LC{(\\a.\a) (\\a.\a) (\\a.\a) (\\a.\a)}| yields: \LC{(\\a.\a) (\\a.\a) (\\a.\a) (\\a.\a)}. But in general, do not re-use variables with different bindings that often within the same expression.
      If you want to change the mixin for each step (local to current group), use |\lcSetDuplicateVariableMixin|. So |\lcSetDuplicateVariableMixin{#1!40!green}| yields: {\lcSetDuplicateVariableMixin{#1!87!green}\LC{(\\a.\a) (\\a.\a) (\\a.\a) (\\a.\a)}}
      If you do not like like the mixin alltogether, you can disable it locally by |\lcDoNotShadeSameVariables| (vs. |\lcDoShadeSameVariables|). This yields:
      {\lcDoNotShadeSameVariables\LC{(\\a.\a) (\\a.\a) (\\a.\a) (\\a.\a)}}
      By changing the tikz-style of |lc@node| you can modify the behavior of highlighted nodes (|#1| is the color).
      The following examples all use |\LC{(\\a.\a (\\b.\b \c)) \a \c}| and make the |\tikzset|-command local to the group:
      \begin{itemize}
         \item |\tikzset{lc@node/.append style={fill=none}}|:\\* {\tikzset{lc@node/.append style={fill=none}}\LC{(\\a.\a (\\b.\b \c)) \a \c}}
         \item |\tikzset{lc@node/.append style={fill=none,text=#1}}|:\\* {\tikzset{lc@node/.append style={fill=none,text=#1}}\LC{(\\a.\a (\\b.\b \c)) \a \c}}
         \item |\tikzset{lc@node/.append style={fill=#1!50!green,text=#1!25!green}}|:\\* {\tikzset{lc@node/.append style={fill=#1!50!green,text=#1!25!green}}\LC{(\\a.\a (\\b.\b \c)) \a \c}}
         \item |\tikzset{lc@node/.append style={opacity=1,circle}}|:\\* {\tikzset{lc@node/.append style={opacity=1,circle}}\LC{(\\a.\a (\\b.\b \c)) \a \c}}
      \end{itemize}

      Similarly, by changing the tikz-style of |lc@abstraction@bracket| you can modify the behavior of brackets for lambdas (|#1| is the color), given they are enabled with |\lcEnableLambdaRanges|.
      The following examples all use |\LC{(\\a.\a (\\b.\b \c)) \a \c}| and make the |\tikzset|-command local to the group:
      \begin{itemize}
         \item |\tikzset{lc@abstraction@bracket/.append style={draw=green}}|:\\* {\tikzset{lc@abstraction@bracket/.append style={draw=green}}\LC{(\\a.\a (\\b.\b \c)) \a \c}}
         \item |\tikzset{lc@abstraction@bracket/.append style={rounded corners=6pt}}|:\\* {\tikzset{lc@abstraction@bracket/.append style={rounded corners=6pt}}\LC{(\\a.\a (\\b.\b \c)) \a \c}}
         \item |\tikzset{lc@abstraction@bracket/.append style={very thick}}|:\\* {\tikzset{lc@abstraction@bracket/.append style={very thick}}\LC{(\\a.\a (\\b.\b \c)) \a \c}}
         \item |\tikzset{lc@abstraction@bracket/.append style={draw=#1!50!pink}}|:\\* {\tikzset{lc@abstraction@bracket/.append style={draw=#1!50!pink}}\LC{(\\a.\a (\\b.\b \c)) \a \c}}
         \item |\tikzset{lc@abstraction@bracket/.append style={opacity=1}}|:\\* {\tikzset{lc@abstraction@bracket/.append style={opacity=1}}\LC{(\\a.\a (\\b.\b \c)) \a \c}}
      \end{itemize}

      Furthermore, you can give parts of your formula readable names:
\begin{lstlisting}[columns=fullflexible]
\LC{(\\a[a].\a[a2] \b \c) \mark{b}{(\\b.\b)}}
\reducemarks{b}{a}[bend right=10]
\reducemarks[south]{a}{a2}[green,bend right=20,-latex]
\end{lstlisting}
      This yields:
      \LC{(\\a[a].\a[a2] \b \c) \mark{b}{(\\b.\b)}}\reducemarks{b}{a}[bend right=10]\reducemarks[south]{a}{a2}[green,bend right=20,-latex]

   By default, the package will automatically enlarge the line skip limit to ensure that your \(\lambda\)-Expressions have enough space (e.g., for presentations). You can disable this with |\lcDoNotStrutLine| (vs. |\lcDoStrutLine|). Here a strutted example: {\lcDoStrutLine\LC{(\\a.\a (\\b.\b \c)) \a \c}}.


   Marks work even within tikzpictures:
   \begin{center}
      \begin{tikzpicture}
         \node[fill=lightgray!80!orange!40!white,rounded corners=2pt, inner sep=5pt] at (0,0) {\RLC{\Y \equiv \\f\.\mark{A}{\(\\x\.\f \(\x \x\)\)} \mark{B}{\(\\x\.\f \(\x \x\)\)}}};
         \reducemarks{B}{A}[bend right=10]
      \end{tikzpicture}
   \end{center}

   With |\lcCreateNewRewritingRule| you can create a new rewriting rule (this is work in progress).
   So with |\lcCreateNewRewritingRule\cons{cons}{\\{x,y,c}\.\c \x \y}| you can now use |\cons| within your \(\lambda\)-calculus expressions. Therefore, \lcCreateNewRewritingRule\cons{cons}{\\{x,y,c}\.\c \x \y}|\LC{\\x.\cons \x}| yields:

   \LC{\\x.\cons \x}.

   There is a starred version as well, that types just the name: |\LC{\\x.\cons* \x}| yields: \LC{\\x.\cons* \x}.

   With this, the faculty function defined like this: |\LC{\Ycomb* (\\{f,n}.\ifelse (\isZero \n) \num{1} (\mult \n (\f (\pred* \n))))}| becomes (scaled down):

   {%
      \tiny\LC{\Ycomb* (\\{f,n}.\ifelse (\isZero \n) \num{1} (\mult \n (\f (\pred* \n))))}%
   }

   With |\lcDisableLambdaRanges| we get:

   {%
      \lcDisableLambdaRanges
      \tiny\LC{\Ycomb* (\\{f,n}.\ifelse (\isZero \n) \num{1} (\mult \n (\f (\pred* \n))))}%
   }

   With |\lcMaximumRewritingRuleExpandDepth| you can control recursive expansion (the default is~\(1\)). So |\lcMaximumRewritingRuleExpandDepth{2}| results in (additionally, |\lcEnableGlow| is active):

   {\lcMaximumRewritingRuleExpandDepth{2}\lcEnableGlow
      \tiny\LC{\Ycomb* (\\{f,n}.\ifelse (\isZero \n) \num{1} (\mult \n (\f (\pred* \n))))}%
   }
\end{document}