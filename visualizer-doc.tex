% Florian Sihler, 2022
% Licensed under MIT License
% https://opensource.org/licenses/MIT
\documentclass[parskip=half,english,numbers=noenddot,footnotes=nomultiple,oneside]{scrartcl}
\errorcontextlines99999

\usepackage[T1]{fontenc}
\usepackage[utf8]{inputenc}
\usepackage{babel}

\usepackage{microtype}
\usepackage[hidelinks]{hyperref}

\usepackage{lmodern}

\usepackage[prefix=]{xcolor-material}
\usepackage{listings}

\usepackage{lc-visualizer}

% working with texcsstyle did not work
\lstdefinelanguage{ltx}{
   literate={\\LC}{\textcolor{purple}{\bfseries\textbackslash LC}}3
   {\\RLC}{\textcolor{purple}{\textbackslash RLC}}4
   {\\lcDisableLambdaRanges}{\textcolor{purple}{\textbackslash lcDisableLambdaRanges}}{21}
   {\\lcEnableLambdaRanges}{\textcolor{purple}{\textbackslash lcEnableLambdaRanges}}{20}
   {\\lcSetColorMixin}{\textcolor{purple}{\textbackslash lcSetColorMixin}}{15}
   {\\lcSetColors}{\textcolor{purple}{\textbackslash lcSetColors}}{11}
   {\\lcMaximumParenthesisRangeDrawDepth}{\textcolor{purple}{\textbackslash lcMaximumParenthesisRangeDrawDepth}}{34}
   {\\a}{\textcolor{gray}{\textbackslash a}}2
   {\\b}{\textcolor{gray}{\textbackslash b}}2
   {\\f}{\textcolor{gray}{\textbackslash f}}2
   {\\x}{\textcolor{gray}{\textbackslash x}}2
   {\\.}{\textcolor{gray}{\textbackslash .}}2
   {\\centerline}{\textcolor{purple}{\textbackslash centerline}}{11}
   {\\Y}{\textcolor{purple}{\textbackslash Y}}2
   {\\equiv}{\textcolor{purple}{\textbackslash equiv}}6
   {\\\\}{\textcolor{teal}{\textbackslash \textbackslash}}2
}
\lstset{basicstyle=\ttfamily,keepspaces,breaklines,language=ltx}
\lstMakeShortInline|
\def\ltx#1{\lstinline/#1/}
\interfootnotelinepenalty=10000

\usepackage{bm}
\lcSetLambda{\bm\lambda}

\title{The \texttt{lc-visualizer} package}
\author{%
	\texorpdfstring{Florian Sihler\medskip\\[-2mm]
		{\scriptsize\url{https://github.com/EagleoutIce/latex-lambda-calculus-visualizer}}%
	}{Florian Sihler}}
\date{Version v1.0 \textendash{} 2022/10/28}


\begin{document}
   \maketitle

	\texttt{lc-visualizer} is a simple package to get highlighting for lambda-calculus expressions.
   Issuing |\LC{(\\a.\\b.\a (\b \\b.\b \a)) a c}| yields the following:

   \LC{(\\a.\\b.\a (\b \\b.\b \a)) a c}

   There is also |\RLC|, a variant that does not do any catcode magic and that can be used inside other macros. So |\centerline{\RLC{\Y \equiv \\f\.\(\\x\.\f (\x \x\)\) \(\\x\.\f \(\x \x\)\)}}| yields:

   \centerline{\RLC{\Y \equiv \\f\.\(\\x\.\f \(\x \x\)\) \(\\x\.\f \(\x \x\)\)}}

   \section{Shortcut and customization options}
      You can pass on multiple letters to a lambda-abstraction: |\LC{\\{a,b}.\b \a}| yields: \LC{\\{a,b}.\b \a}.

      With |\lcDisableLambdaRanges| you can disable the brackets drawn around to signal the abstraction ranges of lambdas locally (or re-enable with |\lcEnableLambdaRanges|):
      |{\lcDisableLambdaRanges\LC{((\\a.\a) \b \a)}}| yields: {\lcDisableLambdaRanges\LC{((\\a.\a) \b \a)}} (vs. \LC{((\\a.\a) \b \a)}).

      With |\lcMaximumParenthesisRangeDrawDepth| you can (similarly to the effects of the |\lcDisableLambdaRanges| macro) allow to draw brackets for parenthesis.
      However, being disabled by default you can no configure the maximum nesting-level to draw this brackets for.
      Writing |{\lcDisableLambdaRanges\lcMaximumParenthesisRangeDrawDepth{2}\LC{((\\a.\a) \b \a)}}| yields: {\lcDisableLambdaRanges\lcMaximumParenthesisRangeDrawDepth{2}\LC{((\\a.\a) \b \a)}} (vs. {\lcDisableLambdaRanges\LC{((\\a.\a) \b \a)}}).

      If you do not like the colors you can change their cycle locally with |\lcSetColors|: |{\lcSetColors{teal,lime}\LC{((\\a.\a) \\b.\b \a)}}| yields: {\lcSetColors{teal,lime}\LC{((\\a.\a) \\b.\b \a)}} (vs. \LC{((\\a.\a) \\b.\b \a)}).

      With |\lcSetColorMixin| you can (locally) change the way colors are mixed in the background highlights (with |#1| being the actual color selected): |{\lcSetColorMixin{#1!70!white}\LC{((\\a.\a) \\b.\b \a)}}| yields the output: {\lcSetColorMixin{#1!70!white}\LC{((\\a.\a) \\b.\b \a)}} (vs. \LC{((\\a.\a) \\b.\b \a)}).

      The package will try its best to keep colors in sync and recognizable:
      |\LC{(\\a.\a) (\\a.\a) (\\a.\a) (\\a.\a)}| yields: \LC{(\\a.\a) (\\a.\a) (\\a.\a) (\\a.\a)}. But in general, do not re-use variables with different bindings that often within the same expression.
   \end{document}
